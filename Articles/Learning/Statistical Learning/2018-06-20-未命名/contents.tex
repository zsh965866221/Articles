%# -*- coding:utf-8 -*-

模型的假设空间为$f\in \mathcal{F}$,损失函数为$\ell(f,X,Y)$,输入输出的联合概率分布为$Pr(X,Y)$,可以是条件概率分布或者决策函数,则损失函数在整个输入输出空间上的期望为
\begin{equation}\label{Eq: exp loss}
\begin{split}
  R_{\textrm{exp}}(f) &= \mathrm{E}_{P_{XY}}[\ell(f,X,Y)] \\
   &= \int_{X\times Y}\ell(f,x,y)Pr(x,y)\mathrm{d}x\mathrm{d}y
\end{split}
\end{equation}
$R_{\textrm{exp}}$叫做期望风险,但由于联合概率分布$Pr(X,Y)$是未知的是需要进行学习估计的。现有训练数据集$\mathcal{D}$,根据该训练数据集会得到一个数据集的平均损失叫做经验风险
\begin{equation}
\begin{split}
    R_{\textrm{emp}}(f) & = \frac{1}{N}\sum_{i=1}^N \ell(f,X,Y)
\end{split}
\end{equation}
根据大数定理,当样本容量$N$趋于无穷时,经验风险$R_{\textrm{exp}}$ 趋于期望风险$R_{\textrm{emp}}$,但现实生活中训练集的样本容量不是无穷的,故在样本容量有限的情况下经验风险与期望风险的差异是多少,是怎么定义的,这就是可学习问题的泛化边界。

\begin{tcolorbox}[title=强]
    This is a \textbf{tcolorbox} with title.
    \tcblower
    Here, you see the lower part of the box.
\end{tcolorbox}

\section{条件均值}
Eq. (\ref{Eq: exp loss})可以看成是关于$f$的能量泛函,我们要做的就是最小化该能量泛函,得到最优的$f$,故把期望误差风险转换成条件概率的形式
\begin{equation}\label{Eq: exp loss conditional}
\begin{split}
  R_{\textrm{exp}}(f) &= \int_{X\times Y}\ell(f,x,y)Pr(x,y)\mathrm{d}x\mathrm{d}y \\
    & = \iint\ell(f,x,y)p(y\mid x)\mathrm{d}y \cdot p(x)\mathrm{d}x \\
    & = \int E_{Y\mid x}[\ell(f,x,Y)]\cdot p(x)\mathrm{d}x \\
    & = E_X[E_{Y\mid X}[\ell(f,X,Y)]]
\end{split}
\end{equation}
对其求偏导
\begin{equation}\label{Eq: partial derv of Exp loss}
\begin{split}
  \frac{\delta R_{\textrm{exp}}}{\delta f} &= \int \frac{\delta \ell(f,x,y)}{\delta f}p(y\mid x)\mathrm{d}y \cdot p(x) \\
    & = E_{Y\mid x}[\frac{\delta \ell(f,x,Y)}{\delta f}]p(x)
\end{split}
\end{equation}
损失函数取方差$\ell(f,X,Y) = (Y-f(X))^2$,
\begin{equation}
\begin{split}
  \frac{\delta R_{\textrm{exp}}}{\delta f} = \int 2(f(x)-y) p(y\mid x) \mathrm{d}y \cdot p(x) = 0
\end{split}
\end{equation}
整理得
\begin{equation}
\begin{split}
    \int f(x)p(y\mid x)\mathrm{d}y &= \int yp(y\mid x)\mathrm{d}y \\
    f(x) &= E_{Y\mid x}[Y] \\
        & = E[Y\mid X=x]
\end{split}
\end{equation}
因此,$Y$的最优的预测是点$x$处的条件均值。对$f(x)$的估计变成了对x点处的条件概率的估计(同联合概率$P(X,Y)$相同,该条件概率也是未知的)。这个地方需要提一下生成模型和概判别模型的区别,生成模型需要得到完整的联合概率分布表示(如混合高斯模型GMM),而判别模型只需要得到条件概率。如何根据现有的观测数据估计出每一个$x$点处的条件概率分布。近邻算法根据测试点$\hat{x}$附近的点来估计该条件分布。《The Elements of Statistical Learning》($P_{19}$)介绍线性回归模型$f(x)\approx x^T\beta$也是按照该原则的。这是基于模型的,则其有模型的假设空间$\mathcal{F}$,其要符合模型定义的约束,在这里线性回归模型的约束为线性超平面。

 
这种估计仍然依据大数定理根据训练集对条件分布进行估计。

\section{泛化边界}
